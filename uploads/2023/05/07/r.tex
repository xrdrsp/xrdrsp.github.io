\RequirePackage{filecontents}
\begin{filecontents*}{\jobname.bib}
@book{neuhaus,
    Title = {论钢琴表演艺术},
    Subtitle = {一个教师的随笔},
    Author = {Neuhaus, H.},
    Isbn = {9787103008720},
    Year = {1963},
    Publisher = {人民音乐出版社},
    Translator = {汪启璋, 吴佩华}
}

@online{bunin,
    Title = "Stanislav Bunin",
    Subtitle = "Wikipedia, The Free Encyclopedia",
    url = "https://en.wikipedia.org/wiki/Stanislav_Bunin",
    addendum = "(retrieved: Apr. 30, 2023)"
}

@online{chopincompetition,
    Title = "International Chopin Piano Competition",
    Subtitle = "Wikipedia, The Free Encyclopedia",
    url = "https://en.wikipedia.org/wiki/International_Chopin_Piano_Competition",
    addendum = "(retrieved: Apr. 30, 2023)"
}

@online{vancliburn,
    Title = "Van Cliburn International Piano Competition",
    Subtitle = "Wikipedia, The Free Encyclopedia",
    url = "https://en.wikipedia.org/wiki/Van_Cliburn_International_Piano_Competition",
    addendum = "(retrieved: Apr. 30, 2023)"
}

@online{lisch,
    Title = "The Leschetizky Legacy",
    url = "https://www.leschetizky.org/about/about-theodor-leschetizky/",
    addendum = "(retrieved: Apr. 30, 2023)"
}

@thesis{liyi,
    Title = {肖邦《f 小调幻想曲》Op. 49 演奏版本之比较研究},
    Author = {王怡},
    institution = {武汉音乐学院},
    year = {2015}
}

@online{workchopin,
    Title = "List of works by Frédéric Chopin",
    url = "https://imslp.org/wiki/List_of_works_by_Fr\%C3\%A9d\%C3\%A9ric_Chopin"
    addendum = "(retrieved: Apr. 30, 2023)"
}

@book{wier,
    Title = {The Piano},
    Author = {Wier, Albert E.},
    Publisher = {New York: Longmans, Green and Cio},
    Year = {1941}
}

@thesis{cwu,
    Title = {An Analysis of the Thematic Structure of Chopin’s Polonaise-Fantaisie, Opus 61},
    Author = {Bull, Thomas W.},
    institution = {Central Washington University}
    Year = {1961}
}

@book{bio,
    Title = {肖邦传},
    Author = {Gavoty, B.},
    Translator = {张雪},
    Year = {2021},
    Publisher = {上海人民出版社},
    Isbn = {9787208167414}
}

@online{fio,
    Title = "SERGIO FIORENTINO DISCOGRAPHY",
    url = "http://www.fortepianos.org/elumpe/SFDiscography.html",
    addendum = "(retrieved: Apr. 30, 2023)"
}

@online{imslp,
    Title = "IMSLP: About",
    url = "https://imslp.org/wiki/IMSLP:About"
    addendum = "(retrieved: May 2, 2023)"
}

@online{49,
    Title = "Fantaisie, Op.49 (Chopin, Frédéric)",
    url = "https://imslp.org/wiki/Fantaisie,_Op.49_(Chopin,_Fr\%C3\%A9d\%C3\%A9ric)",
    addendum = "(retrieved: May 2, 2023)"
}

@online{61,
    Title = "Polonaise-Fantaisie, Op.61 (Chopin, Frédéric)",
    url = "https://imslp.org/wiki/Polonaise-fantaisie,_Op.61_(Chopin,_Fr\%C3\%A9d\%C3\%A9ric)",
    addendum = "(retrieved: May 2, 2023)"
}
\end{filecontents*}

\documentclass[12pt]{article}
\usepackage{ctex}
\usepackage{multirow}
\usepackage{amsmath}
% \usepackage[T1]{fontenc}
% \usepackage[utf8]{inputenc}
\usepackage{inputenc}
\usepackage{authblk}
\usepackage{makecell}
\usepackage{hyperref}
\usepackage{fancyhdr}
\usepackage{enumerate}
\usepackage{multicol}
\usepackage{indentfirst}
\usepackage{geometry}
\usepackage{csquotes}
\usepackage{array}
\usepackage{graphicx}
\usepackage{appendix}
\usepackage{tikz}
\usepackage{float}
\usepackage{longtable}
\usepackage{tabularx}

\usepackage[
backend=bibtex
]{biblatex}
% \addbibresource{biblio.bib}
\addbibresource{\jobname.bib}

% \geometry{left=1cm, right=1cm, top=1.5cm, bottom=1.5cm}

% \noindent 中文字体(默认宋体)\\
% \fangsong 中文字体(仿宋) \songti 中文字体(宋体) \lishu 中文字体(隶书) \heiti 中文字体(黑体)\\
% \CJKfamily{zhkai} 中文字体(楷书) \CJKfamily{zhyou} 中文字体(幼圆) \CJKfamily{zhyahei} 中文字体(微软雅黑)\\

% 三、小三、四、小四、五:16、15、14、12、10.5

\title{\heiti \fontsize{18}{25}\selectfont 对一些肖邦大型钢琴独奏作品演绎录音的研究\\
\fontsize{15}{25}\selectfont ——以《f 小调幻想曲》与《降 A 大调幻想波兰舞曲》为例}
\author{高源\thanks{\href{mailto:xrdrsp@gmail.com}{xrdrsp@gmail.com}}}
\affil{宁波效实中学 D2504 班}
\date{}


\begin{document}
    \maketitle
    \begin{abstract}
        \noindent 弗雷德里克·弗朗索瓦·肖邦(法语:Frédéric François Chopin;波兰语:Fryderyk Franciszek Chopin)创作的《f 小调幻想曲,作品 49》(\textit{Fantaisie in f Minor, Op. 49})与《降 A 大调幻想波兰舞曲》(\textit{Polonaise-Fantaisie in A Flat Major, Op. 61})是其创作生涯中两部重要的大型作品。它们在钢琴独奏音乐会上常见,亦有许多演绎录音的版本留存和发布。本文从演奏者的角度出发,主要通过乐谱和一些演奏录音的分析,根据作曲家的乐谱标记、古今演奏录音等等,获得这两部具有代表性的肖邦作品的一些演绎风格、特征、规则;并希望对更好地演绎肖邦的音乐产生一些促进作用。

        \noindent \textbf{关键词}:钢琴演奏\ \ 肖邦\ \ 幻想曲\ \ 幻想波兰舞曲\ \ 录音
    \end{abstract}

    \newpage
    \tableofcontents
    \newpage

    \setlength\parindent{2em}
    % \section{\heiti \fontsize{16}{19.2}\selectfont 标题}
    % 这是一段文字。这是一段文字。这是一段文字。\footnote{这是一个脚注。}这是一段文字。这是一段文字。这是一段文字。这是一段文字。这是一段文字。这是一段文字。这是一段文字。这是一段文字。这是一段文字。这是一段文字。这是一段文字。这是一段文字。这是一段文字。这是一段文字。这是一段文字。这是一段文字。这是一段文字。这是一段文字。这是一段文字。这是一段文字。这是一段文字。这是一段文字。这是一段文字。这是一段文字。这是一段文字。这是一段文字。这是一段文字。

    \section{\heiti \fontsize{16}{19.2}\selectfont 引言}
    \subsection{\heiti \fontsize{15}{18} \selectfont 关于作曲家:弗雷德里克·弗朗索瓦·肖邦}
    弗雷德里克·弗朗索瓦·肖邦(1810—1849)是伟大的浪漫主义波兰作曲家。其一生大部分作品是为钢琴而作的,从《G 大调波兰舞曲,B. 1》(\textit{Polonaise in G Major, B. 1},创作于 1817 年)到《f 小调玛祖卡,作品 68 之 4(遗作)》(\textit{Mazurka in f Minor, Op. 68 No. 4 (posth.)},创作于 1849 年);只有少数作品不止是为钢琴创作的,如作品 2、11、13、14、21 和 22 \footnote{特别地,指作品 22 中的华丽的大波兰舞曲(\textit{Grande Polonaise Brillante})。}中运用了钢琴和管弦乐队,作品 3、8\footnote{作品 8 是钢琴三重奏,还运用了小提琴。}、65 与 B. 70 中使用了钢琴和大提琴,等等。

    肖邦对钢琴音乐的贡献是杰出的。他对夜曲、玛祖卡、前奏曲等多种体裁进行革新,在《第二钢琴奏鸣曲,作品 35》(\textit{Piano Sonata in b Flat Minor, Op. 35})等许多作品中成为结构、和声、织体等多方面创新的先锋派。他的主要作品有《波兰舞曲》(17 首,\textit{Polonaises})、《玛祖卡》(57 首,\textit{Mazurkas})、《前奏曲,作品 28》(\textit{Preludes, Op. 28})\footnote{除了作品 28 外,肖邦还写过两首前奏曲:《升 c 小调前奏曲,作品 45》(\textit{Prelude in c Sharp Minor, Op. 45})和《降 A 大调前奏曲,B. 86》(\textit{Prelude in A Flat Major, B. 86})。}、《叙事曲,作品 23、38、47、52》(\textit{Ballades, Opp. 23, 38, 47, 52})、《谐谑曲,作品 20、31、39、54》(\textit{Scherzos, Opp. 20, 31, 39, 54})、《奏鸣曲,作品 4、35、58、65》(\textit{Sonatas, Opp. 4, 35, 58, 65})\footnote{其中,前三首是钢琴奏鸣曲,最后一首是大提琴奏鸣曲。}、《圆舞曲》(14 首,\textit{Waltzes})、《练习曲》(27 首,\textit{Etudes})、《夜曲》(19 首,\textit{Nocturnes})、《协奏曲,作品 11、21》(\textit{Piano Concertos, Opp. 11, 21})。\cite{workchopin}

    \subsection{\heiti \fontsize{15}{18} \selectfont 关于我们的问题}

    自留声机的发明开始,录音技术飞速进步。这使得二十世纪上半叶以来,从“蜡桶”、“纸带”到声学的肖邦作品录音层出不穷;其中便包含了钢琴演奏录音史上为人称道的“黄金时代”,那时钢琴制造水平、钢琴演奏水平达到了较高的程度。“黄金时代”主要以二十世纪上半叶众多钢琴演奏家为代表。莱谢蒂茨基(Theodor Leschetizky;波兰语:Teodor Leszetycki)及其众多学生,如马克·汉博格(Mark Hambourg)等虽未留下十分充足的录音,但也使我们一瞥黄金时代钢琴演奏的旨趣;类似的还有涅高兹(Neuhaus)家族,如海因里希·涅高兹(Heinrich Neuhaus)、斯坦尼斯拉夫·涅高兹(Stanislav Neuhaus),他们为我们留下了较为清晰的声学录音,也有一些演奏理论的著作传世 \cite{neuhaus}。其余的还有如 Lazare-Lévy(“法国钢琴学派”代表)、珀西·格兰杰(Percy Aldridge Grainger)、约瑟夫·霍夫曼(波兰语:Józef Kazimierz Hofmann)等等。而对《f 小调幻想曲》,Jerzy Żurawlew、列奥·希洛塔(Leo Sirota)、Victor Gille、Myra Hess 等等演奏者留下了伟大的录音。对《降 A 大调幻想波兰舞曲》,Sergio Fiorentino、Marcel Ciampi 等等的演绎是极有说服力的。

    近几十年来\footnote{需要说明的是,此处“黄金时代”并不需要准确的时间区段;本文中将录音按是否属于“黄金时代”划分,只是按照年代和被忽略程度大致分类,以便于接下来的讨论。},录音技术、信息技术达到新的高度,钢琴演奏录音中对演奏者表现的细节体现得更加准确和细腻,录音编辑工作更加精确:当代演奏者在这一点获得了便利。对《f 小调幻想曲》,具有代表性的是米哈伊尔·普列特涅夫(Mikhail Vasilievich Pletnev)与克里斯蒂安·齐默尔曼(Krystian Zimerman)的录音。对《降 A 大调幻想波兰舞曲》,也有普列特涅夫、斯坦尼斯拉夫·布宁(Stanislav Stanislavovich Bunin)\cite{bunin}\footnote{布宁,即斯坦尼斯拉夫·涅高兹之子,也属于涅高兹家族。}、傅聪等等的优秀录音。

    将录音按照时间莽撞地分成两类,在欣赏上实在不是一个明智之举;这里只是出于分析需要。显然,当代录音音质显著优于二十世纪早中期的;在世的演奏者大多具有钢琴比赛获奖经历\footnote{二十世纪早中期,钢琴比赛事业逐渐发展起来,如肖邦国际钢琴比赛\cite{chopincompetition}、范·克莱本国际钢琴比赛\cite{vancliburn}。},仍具有商业影响力:在“学院派”等等多数群体中,当代演奏者的录音更加受到喜爱,而早期的录音,甚至“黄金时代”录音处于被忽略的位置。但当代录音与“黄金时代”录音间有较大的差别,这一点将在正文展开分析;“黄金时代”钢琴家具有离作曲家更近的师承关系,这是不可否认的\footnote{如,莱谢蒂茨基是车尔尼(Carl Czerny)的学生;后者是贝多芬(Ludwig van Beethoven)的学生。\cite{lisch}}。尝试重新分析那些被忽略的演奏者的录音,与当代著名的录音进行对比;并在这基础上对现在的钢琴演奏提出启发,可能对当代演奏者有较大益处。但是,现有的公开文献中,对于钢琴演奏录音版本进行实践性研究,而非停留在乐谱或历史分析的文字较为少见 \cite{liyi};其中缺少针对早期演奏、“黄金时代”演奏的文献。本文写作目的即在基于对少量肖邦作品与钢琴演奏录音的实践性比较和分析,获知一些钢琴演奏的旨趣,特别是当代钢琴演奏相比“黄金时代”钢琴演奏缺失的旨趣,并希望对更好地演绎肖邦的音乐产生一些促进作用。

    \section{\heiti \fontsize{16}{19.2}\selectfont 对象}
    \subsection{\heiti \fontsize{15}{18}\selectfont 《f 小调幻想曲》}

    《f 小调幻想曲》,作于 1840—1841 年,题献给苏索公主(法语:Madame la Princesse Catherine de Souzzo)。这是一首单乐章大型作品,演奏时长约 10 余分钟。肖邦只创作过一首钢琴独奏幻想曲(Fantaisie)。贝尔纳·加沃蒂(Bernard Gavoty)对这部作品评价道:
    \begin{itemize}
        \item[] \textit{……在几个肖邦给予转调和加速的即兴段落后,情绪便随之不停转换。至此,我们进入主导动机,转入一个暗色调的主题,直至出现一个魅力十足、激动人心的句子,这暗色的主题才顿现辉煌:光明战胜黑暗。乐曲展开,节奏摇曳,钢琴技巧变换不已。终于一支高贵的骑士风格歌咏骤然而降,伴随着左手拨弹连奏。自此,肖邦无需再使用引导要素,他采用其他动机,直入一个绵延的慢板,虽只一瞬,却带来沉思、冥想、宁静。这时重复引子后的第一个乐段,作临时变调,接下来是一系列的转调,并突然结束在一个变格终止上。}\cite{bio}
    \end{itemize}
    
    虽然题目和体裁是“幻想曲”,仿佛该作品写得较为自由,其结构却异常严谨规整\footnote{肖邦其他类似的大型作品(如叙事曲、谐谑曲、一些波兰舞曲)常采用多种曲式混合的写法,结构不易令人发现。}。大体上讲,幻想曲的结构大概可以由奏鸣曲式解释,如下表所示:
    \newpage

    \begin{center}
    \heiti \fontsize{10.5}{12.6}\selectfont 表 2.1 \ \ 《f 小调幻想曲》结构
    \end{center}
    \begin{longtable}{|l|l|l|l|l|}
    % \centering
    % \begin{longtable}{}
    \hline
        记号 & 小节 & 结构 & 内容 & 调性 \\ \hline
        ~ & 1—20 & 序奏 & 序奏 1 & f 小调 \\ 
        ~ & 21—42 & ~ & 序奏 2 & f 小调 \\ 
        ~ & 43—67 & 呈示部 & 主题 A & f 小调 \\ 
        ~ & 68—84 & ~ & 主题 B & f 小调 — 降 A 大调 \\ 
        ~ & 85—92 & ~ & 主题 C & 降 A 大调 \\ 
        \textit{Marcia (Grave)}\footnote{手稿如此。一些版本的乐谱中改为 \textit{Tempo di marcia}。} & 93—126 & ~ & 主题 D & c 小调 \\ 
        ~ & 127—143 & ~ & 主题 E & 降 E 大调 \\ 
        ~ & 143—154 & 展开部 & 主题 A & E 大调 \\ 
        ~ & 155—171 & ~ & 主题 B & c 小调 — 降 E 大调 — 降 G 大调 \\ 
        ~ & 172—179 & ~ & 主题 C & 降 E 大调 \\ 
        ~ & 180—198 & ~ & 主题 A & 降 E 大调 — 降 G 大调 \\ \hline
        \textit{Lento, Sostenuto} & 199—222 & 中段 & 主题 F & B 大调 \\ \hline
        \multirow{6}{*}{\textit{Tempo primo}} & 223—234 & 再现部 & 主题 A & 降 b 小调 — 降 C 大调 \\ 
        ~ & 235—251 & ~ & 主题 B & 降 b 小调 — 降 D 大调 \\ 
        ~ & 252—259 & ~ & 主题 C & 降 D 大调 \\ 
        ~ & 260—293 & ~ & 主题 D & f 小调 — 降 A 大调 \\ 
        ~ & 294—310 & ~ & 主题 E & 降 A 大调 \\ 
        ~ & 310—319 & 尾声 & 主题 A & a 小调 \\ \hline
        \textit{Adagio sostenuto} & 320—321 & 尾声 & 主题 F & 降 A 大调 \\ \hline
        \textit{Allegro assai} & 322—332 & 尾声 & 主题 A & 降 A 大调 \\ \hline
        % \end{longtable}
    \end{longtable}


    \subsection{\heiti \fontsize{15}{18}\selectfont 《降 A 大调幻想波兰舞曲》}
    《降 A 大调幻想波兰舞曲》,作于 1846 年,题献给 A Madame A. Veyret,演奏时长约 10 余分钟。该作品出版后由于各种原因得到包括李斯特(匈牙利语:Liszt Ferenc)的众人批评,但终究获得了认可:Albert Ernest Wier 将其评为史上最伟大的钢琴作品之一 \cite{cwu} \cite{wier},它也在世界各地频繁演奏。肖邦曾在一封家信中提到:“他正为一支新曲的题目而困惑” \cite{bio}。这首作品最终拥有《幻想波兰舞曲》的名字,但带有波兰舞曲特色的节奏在曲中只是偶尔出现。阿尔弗雷德·科尔托(Alfred Cortot)评价道:
    \begin{itemize}
        \item[] \textit{……之后,随着略微挣脱梦幻的非现实性,形成一种激越递增的声响效果。这一刻,出现了让作者自己都觉眩晕的激动人心的幻觉,一种充满了荣耀的幻象:胜利的波兰完成了自己的使命。} \cite{bio}
    \end{itemize}

    《幻想波兰舞曲》结构如下表所示:

    \begin{center}
    \heiti \fontsize{10.5}{12.6}\selectfont 表 2.2 \ \ 《降 A 大调幻想波兰舞曲》结构
    \end{center}
    \begin{longtable}{|l|l|l|l|}
    \hline
        记号 & 小节 & 内容 & 调性 \\ \hline
        \multirow{5}{*}{\textit{Allegro Maestoso}} & 1—23 & 序奏 & 降 A 大调 \\ 
        & 24—65 & 主题 A & 降 A 大调 \\ 
        & 66—91 & 主题 B & 降 A 大调 — E 大调 \\ 
        & 92—115 & 主题 A & 降 B 大调 \\ 
        & 116—147 & 主题 C & 降 B 大调 — D 大调 — 升 F 大调 \\ \hline
        \multirow{4}{*}{\textit{Poco più lento}} & 148—180 & 主题 D & B 大调 \\ 
        & 180—213 & 主题 C & 升 g 小调 — B 大调 \\ 
        & 214—215 & 序奏 & B 大调 — D 大调 — A 大调 — C 大调 \\ 
        & 216—225 & 主题 C & f 小调 \\ \hline
        \multirow{3}{*}{\textit{Tempo }I} & 226—241 & 连接部 & 降 A 大调 — B 大调 \\ 
        & 242—253 & 主题 A & 降 A 大调 — B 大调 — 降 A 大调 \\
        & 254—288 & 主题 D & 降 A 大调 \\ \hline
    \end{longtable}

    \section{\heiti \fontsize{16}{19.2}\selectfont 分析}
    作为演奏者,除了把握作品曲式结构(应用于分句)之外,对于和声、织体等的整体分析是必要的;现在由于客观条件限制,只在必要时作具体的、单一的分析。
    \subsection{\heiti \fontsize{15}{18}\selectfont 概况}
    现选取一些录音如下\footnote{对于每首曲目,所选前两个录音是“黄金时代”录音,最后一个录音是当代钢琴家录音。本文所选录音可以在附录 A 中获取详细信息。}:

    \begin{itemize}
    \item 《f 小调幻想曲》
        \begin{enumerate}
            \item Jerzy Żurawlew 于 1955 年的录音;
            \item Myra Hess 于 1937 年的录音;
            \item Evgeny Kissin 于 1993 年的录音;
        \end{enumerate}
    \item 《降 A 大调幻想波兰舞曲》
        \begin{enumerate}
            \item Sergio Fiorentino 于 1959 年的录音;
            \item Marcel Ciampi 于 1952 年的录音;
            \item 普列特涅夫于 2022 年的录音。
        \end{enumerate}
    \end{itemize}

    \subsection{\heiti \fontsize{15}{18}\selectfont 从整体上考虑:结构与分句}
    本文选取的两首作品均符合一特点:表演者演绎自由度较高;这首先体现于演奏时长。对于《f 小调幻想曲》,较快的演绎出自 Victor Gille 等演奏者,需要 11 分 20 秒左右;“黄金时代”钢琴家的录音也都一般需 11—12 分钟。进入本世纪,演奏时长整体上越来越长,13—14 分钟的演绎成为常见的情况,较为极端的还有 Ivo Pogorelich 的 16 分 11 秒\footnote{来自 Sony 公司发行的唱片“Chopin”。}。由于客观条件限制,本文无法放入更多演绎的分析;但从已纳入本文的录音对比中可略见一斑。


    \begin{center}
    \heiti \fontsize{10.5}{12.6}\selectfont 表 3.1 \ \ 《f 小调幻想曲》演奏时长统计
    \end{center}
    \begin{longtable}{|l|l|l|}
    \hline
        Jerzy Żurawlew & Myra Hess & Evgeny Kissin \\ \hline
        12’ 25” & 12’ 48” & 14’ 01” \\ \hline
    \end{longtable}

    \subsubsection{\heiti \fontsize{14}{16.8}\selectfont 《f 小调幻想曲》序奏}

    \begin{center}
    \heiti \fontsize{10.5}{12.6}\selectfont 表 3.2 \ \ 《f 小调幻想曲》序奏演奏时长统计
    \end{center}
    \begin{longtable}{|l|l|l|}
    \hline
        Jerzy Żurawlew & Myra Hess & Evgeny Kissin \\ \hline
        2’ 30” & 2’ 53” & 3’ 19” \\ \hline
    \end{longtable}

    % \begin{center}
    % \heiti \fontsize{10.5}{12.6}\selectfont 表 3.3 \ \ 《f 小调幻想曲》中段演奏时长统计
    % \end{center}
    % \begin{longtable}{|l|l|l|}
    % \hline
    %     Jerzy Żurawlew & Myra Hess & Evgeny Kissin \\ \hline
    %     1’ 42” & 1’ 45” & 1’ 51” \\ \hline
    % \end{longtable}

    这样的情况并不是个例。“黄金时代”录音整体上是快于当代演奏的。值得注意的是,肖邦在序奏开头写出的标记是 \textit{Marcia (Grave)},即进行曲。虽然序奏有与葬礼进行曲类似的节奏,但肖邦从未表明序奏是葬礼进行曲。Evgeny Kissin 的演奏不同于所选的其他演奏,显得慢而沉,是带有葬礼进行曲的倾向的。Jerzy Żurawlew 和 Myra Hess 的演奏则体现出普通的进行曲特征。

    \begin{center}\includegraphics[scale=0.45]{1.png}\\
    \heiti \fontsize{10.5}{12.6}\selectfont 图 3.1 \ \ 谱例 1\footnote{本文所选乐谱可以在附录 B 中获取详细信息。}
    \end{center}

    面对两段序奏的连接,三个录音中给出不同的方法。


    \begin{center}\includegraphics[scale=0.45]{2.png}\\
    \heiti \fontsize{10.5}{12.6}\selectfont 图 3.2 \ \ 谱例 2:两段序奏的连接
    \end{center}

    Evgeny Kissin 在第 17 小节(图 3.2 中第 4 小节)第 3 拍即开始渐强,到第 19 小节达到最高点。之后,在该小节 1—4 拍突然进行渐弱,其中伴随着音色的突变\footnote{此处指在演奏者突出的一段完整的动态变化或线条中产生音色的突变:这显然破坏了线条本身。};在该小节最后 0.25 拍音量突然升高,进行第二次渐弱,立刻进入第二段序奏。特别地,在第 18 小节第 3 拍,与前后均形成矛盾地突出了最高音(前后均为厚实的和弦)。

    Myra Hess 在第 17 小节直接以较强的音量进入,并未在句中加入过多动态对比;音量向上达到同样位置的最高点。在 20 小节而非 19 小节,Myra Hess 渐弱且做出音色变化,最后一拍后出现一个短停顿,接入下一段序奏。第 20 小节以后音色统一,过渡显得自然;动态和音色的对比只在两个整句之间出现,而取消了小的线条中的变化,连贯而富有歌唱性。这凸显出其分句的合理性。针对第 18 小节第 3 拍,Myra Hess 在每个音时值上作出精妙的分配而非直接“从下往上冲去”,解决了旋律线厚度和分配的问题。Jerzy Żurawlew 的处理是类似的:动态幅度大、规模大,分句自然且连贯。相比之下,Evgeny Kissin 较为鲁莽。



    \subsubsection{\heiti \fontsize{14}{16.8}\selectfont 《f 小调幻想曲》尾声}


    \begin{center}\includegraphics[scale=0.45]{3.png}\\
    \heiti \fontsize{10.5}{12.6}\selectfont 图 3.3 \ \ 谱例 3:变格终止选段
    \end{center}

    自第 310 小节,乐曲结束再现部高潮,突然进入变格终止。肖邦连续使用 \textit{\textbf{ff}} 等记号,说明该片段乐曲进行激烈的程度。主题 A 接入主题 F 时,乐曲从快板变为慢板,并伴随拍号变化,产生强烈的反差,需要演奏者作出特别的处理。

    Evgeny Kissin 在 316 小节(图 3.3 中第 1 小节)第 2 拍违反乐谱标记做出一个突弱,在之后通过一个长渐强到达第 319 小节。他的音色、响度达到统一,直到第 321 小节第 2 拍的突弱和音色变化。这暗示着 Evgeny Kissin 的分句并未按照乐谱进行。虽然如此,第 320 小节第 2 拍踏板的松开是令人迷惑的:这样,Evgeny Kissin 的分句是不明晰的。此外,第 321 小节第 2 拍导入音色、响度均相同的第 3 拍时突然的停顿更加破坏了旋律的线条。在之后的华彩中也可发现类似的情况。

    Jerzy Żurawlew 做出令人惊叹的处理。第 316 小节至 318 小节的重音完全按照乐谱标记处理,增加了旋律的厚度。此外,他做出了长渐弱,直到第 319 小节的和声。此后声部逐渐分离,各声部动态和音色上作出变化,直至第 320 小节自下而上,导入最高音的旋律。此时 Jerzy Żurawlew 违反乐谱标记,响度并未达到 \textit{\textbf{ff}};但旋律的音色已经与之前形成鲜明的对比,变得温暖、厚重。分句明确,体现 Jerzy Żurawlew 精妙的分句和高超的手指技术。

    \subsubsection{\heiti \fontsize{14}{16.8}\selectfont 《降 A 大调幻想波兰舞曲》连接部}
    《降 A 大调幻想波兰舞曲》是肖邦晚期作品,结构甚为自由,由一些主题和动机支撑全曲。普列特涅夫和 Sergio Fiorentino 都对乐谱作了较大改动。


    \begin{center}\includegraphics[scale=0.45]{4.png}\\
    \heiti \fontsize{10.5}{12.6}\selectfont 图 3.4 \ \ 谱例 4:部分连接部
    \end{center}
    \begin{center}\includegraphics[scale=0.5]{5.png}\\
    \heiti \fontsize{10.5}{12.6}\selectfont 图 3.5 \ \ 谱例 5:第 226 小节原旋律声部
    \end{center}
    \begin{center}\includegraphics[scale=0.5]{6.png}\\
    \heiti \fontsize{10.5}{12.6}\selectfont 图 3.6 \ \ 谱例 6:Sergio Fiorentino 更改后的第 226 小节旋律声部
    \end{center}

    Sergio Fiorentino 作了如图所示更改,使得旋律向上的趋势更为强烈。第 226—228 小节,Sergio Fiorentino 进行一次渐强,特别突出了第 228—229 小节左手的波兰舞曲特殊节奏。这在结构上与之前主题的呈现产生呼应;左手断奏与右手的长连奏形成对比,与主题 A 第一次呈现、该节奏第一次出现时的连奏形成对比,进一步突出了该作品作为波兰舞曲的特性。

    \begin{center}\includegraphics[scale=0.2]{11.png}\\
    \heiti \fontsize{10.5}{12.6}\selectfont 图 3.7 \ \ 谱例 7:波兰舞曲体裁的特殊节奏
    \end{center}
    \begin{center}\includegraphics[scale=0.45]{7.png}\\
    \heiti \fontsize{10.5}{12.6}\selectfont 图 3.8 \ \ 谱例 8:相同内容转调重复一次
    \end{center}

    相同内容转调重复一次,第二次渐强。

    \begin{center}
    \includegraphics[scale=0.45]{8.png}\\
    \heiti \fontsize{10.5}{12.6}\selectfont 图 3.9 \ \ 谱例 9:部分连接部
    \end{center}
    \begin{center}
    \includegraphics[scale=0.45]{9.png}\\
    \heiti \fontsize{10.5}{12.6}\selectfont 图 3.10 \ \ 谱例 10:部分连接部,接入主题 A
    \end{center}

    第 240—241 小节,Sergio Fiorentino 加入左手向下八度音,增厚了旋律,使音响效果和动态变化更加突出。大动态、幅度的渐强流畅地引入主题 A 的辉煌再现,其中可以体会严谨的结构构思和演奏者的感情流露。

    \begin{center}
    \includegraphics[scale=0.425]{10.png}\\
    \heiti \fontsize{10.5}{12.6}\selectfont 图 3.11 \ \ 谱例 11:主题 A 再现
    \end{center}

    经向上推动达到的《降 A 大调幻想波兰舞曲》主题 A、D 均为先前出现过的主题再现。从作品整体上讲,这两次再现是相对之前乐曲所有进行的一次动态对比\footnote{作曲家多次使用了 \textit{\textbf{ff}}、\textit{sempre \textbf{ff}}(保持极强)等之前较少出现的记号。},各种动机变换技巧呈现。这样的再现很大程度上巩固了该作品本难以发现的结构,也主要是这样的“回归”使该作品的尾声更能打动人。



    \subsection{\heiti \fontsize{15}{18}\selectfont 音色、声部处理与其他细节处}

    \subsubsection{\heiti \fontsize{14}{16.8}\selectfont 《f 小调幻想曲》歌唱性主题}

    \begin{center}
    \includegraphics[scale=0.45]{12.png}\\
    \heiti \fontsize{10.5}{12.6}\selectfont 图 3.12 \ \ 谱例 12:呈示部主题 B
    \end{center}

    自第 68 小节(图 3.12 中第 2 小节)始,乐曲进入主题 B。这是一个由右手演奏歌唱性旋律的主题,其中可分成两句:前一句有作曲家标记“\textit{agitato}”(激动地);经过左手连奏蓄势和旋律向高处发展,引入后一句更加温暖的歌唱\footnote{即贝尔纳·加沃蒂所评“光明战胜黑暗”处。}(自图 3.13 中第 5 小节始)。此处 Jerzy Żurawlew 对音色的控制体现得较为明显。随着乐曲的进行和性格的变化,Jerzy Żurawlew 在动态处理的同时,流畅地将沉重的音色变为上浮的、剔透的音色。这样的变化是十分吸引人的。\footnote{此处受研究方法限制,无法更进一步地进行科学的分析,只能用一些定语加以概括。那些定语也不应当被赋予严格或唯一的意义;而将那些定语加入本文,是出于研究途径和方向的需要,也是出于分析的实践性的需要(不仅仅停留在乐谱上),因此并不对本文分析的实用性产生影响。}

    \begin{center}
    \includegraphics[scale=0.43]{13.png}\\
    \includegraphics[scale=0.43]{14.png}\\
    \heiti \fontsize{10.5}{12.6}\selectfont 图 3.13 \ \ 谱例 13:呈示部主题 B
    \end{center}

    在主题 B 第二段温暖的旋律中,Jerzy Żurawlew 保留了根音(最低音)的高地位。踏板处理得恰到好处,听觉上左手声部清晰且分离,又与右手优美自然的歌唱结合在一起;此时弱化了动态处理的作用,保持到下一主题的出现。这对于主题性格的塑造是有效的。Myra Hess 的录音中也有类似的处理。Jerzy Żurawlew 和 Myra Hess 在这个主题中创造了一种绚丽、复杂的色彩,并以此打动了听众。

    \subsubsection{\heiti \fontsize{14}{16.8}\selectfont 《f 小调幻想曲》中段}

    \begin{center}
    \includegraphics[scale=0.43]{15.png}\\
    \includegraphics[scale=0.43]{16.png}\\
    \heiti \fontsize{10.5}{12.6}\selectfont 图 3.14 \ \ 谱例 14:中段
    \end{center}

    自第 199 小节(图 3.14 中第 1 小节)始,乐曲进入一个新的主题,以多声部“合唱”的形式呈现。Jerzy Żurawlew 在其中保持了统一的优美音色,并进行了细腻的声部处理。总体上突出了高声部旋律,其气息控制得稳定且自然,弹性速度(\textit{rubato})在细微处体现。Evgeny Kissin 的版本在性格塑造上则显得矛盾、局促不安:整体上采用比 Jerzy Żurawlew 更慢的速度,但弹性速度控制不稳定。第 200 小节第 2 拍明显的抢拍进入使得旋律散乱,第 205 小节最后 0.5 拍进入时拖延——这样的处理是经不起推敲的。它们显得不自然,也不满足以弹性速度演奏旋律的规则;即使这样拖延或抢拍的处理是因为在前后某处节奏、时间分配(timing)的补偿,它们也与弱奏带来的氛围产生了冲突。

    自第 206 小节始,Jerzy Żurawlew 通过动态和音色处理,使得色彩更为静谧。左手弹奏半音阶下行的声部稳定而略带紧张感,与右手旋律在 6 小节中完成一次向上的推动,动态和音色的变化一气呵成;到第 212 小节达到最高点后渐弱,通过少许渐慢和弹性速度的处理,重新回到第一段主题 F 的再现。除了此处外,Jerzy Żurawlew 几乎未进行任何弹性速度处理,节制而自然,对比充分,旋律线条完整。

    \begin{center}
    \includegraphics[scale=0.45]{17.png}\\
    \heiti \fontsize{10.5}{12.6}\selectfont 图 3.15 \ \ 谱例 15:中段
    \end{center}

    自第 215 小节始,Jerzy Żurawlew 进行了细腻的声部处理,适当提高了低声部,特别是左手声部的地位。此时作曲家所写的和声色彩不断转变,这样的声部处理进一步提升了演绎的深度和趣味。

    \subsubsection{\heiti \fontsize{14}{16.8}\selectfont 《降 A 大调幻想波兰舞曲》中段慢板复调}

    与《f 小调幻想曲》中段相同,《降 A 大调幻想波兰舞曲》中段慢板也出现在作品中部,以弱奏、慢板\footnote{《f 小调幻想曲》中为 \textit{Lento, Sostenuto},《降 A 大调幻想波兰舞曲》中为 \textit{Poco più lento}。}呈现,且也为 B 大调。

    \begin{center}
    \includegraphics[scale=0.45]{18.png}\\
    \includegraphics[scale=0.45]{19.png}\\
    \heiti \fontsize{10.5}{12.6}\selectfont 图 3.16 \ \ 谱例 16:中段慢板
    \end{center}

    本文所选三个录音,在中段慢板均做出了极为细腻的音色,其中略有不同。Marcel Ciampi 与 Sergio Fiorentino 的音色显得内敛;普列特涅夫的音色则晶莹剔透。Sergio Fiorentino 在第 148—150 小节的和声预备中进行一次推动,通过声部处理到达第 150 小节第一拍和弦的最高音,以此奠定该段音色上的基调。之后速度处理自由,弹性速度运用广泛,但根音保持稳定。值得注意的是,在第 168 小节第一拍,Sergio Fiorentino 加入了向下八度的 B 音,音响效果发生变化,和声更加厚重。

    \begin{center}
    \includegraphics[scale=0.45]{20.png}\\
    \heiti \fontsize{10.5}{12.6}\selectfont 图 3.17 \ \ 谱例 17:中段慢板
    \end{center}

    三个录音都将声部处理得精当。左手呈现的主题即为序奏中的变奏,且之后将在尾声再现 \cite{cwu},是一个重要部分,三位演奏者都对此给予不同程度的特殊的处理。不同声部的线条时有交叉和配合,但总体上处理中体现的声部分离感强,创造了不同的意境,均为好的演奏。复调的形式下,对于各声部动态、音色等进行更加详细的分析要耗去冗长的空间\footnote{作曲家在该段就有近 30 处针对动态的标记。},此处不再赘述。


    \subsubsection{\heiti \fontsize{14}{16.8}\selectfont 《降 A 大调幻想波兰舞曲》慢板单旋律歌唱}

    \begin{center}
    % \includegraphics[scale=0.45]{21.png}\\
    \includegraphics[scale=0.45]{22.png}\\
    \includegraphics[scale=0.45]{23.png}\\
    \heiti \fontsize{10.5}{12.6}\selectfont 图 3.18 \ \ 谱例 18:慢板中单旋律歌唱段落
    \end{center}

    复调部分之后,乐曲进入了一个单旋律歌唱的段落,是之前曾出现的一个段落的变奏。肖邦在此处并未给出许多动态标记符号。Marcel Ciampi 用一个渐强进入第 189 小节,并在第 192 小节下一句进入时收回,在旋律的反复中形成对比。旋律、歌唱的极高自由度可以赋予演奏者和听众丰富的想象,打动人心。普列特涅夫做出截然相反的处理。整体上其速度慢且自由,以极弱的音量进入第 189 小节,使晶亮的旋律音色被包在左手开放排列的和弦中。在第 191 小节左手声部断奏,第 192 小节稍强,使得主题形象更加多变,意境隽永。另外,在之后几小节中普列特涅夫修饰性地更改了乐谱。

    \begin{center}
    \includegraphics[scale=0.45]{24.png}\\
    \heiti \fontsize{10.5}{12.6}\selectfont 图 3.19 \ \ 谱例 19:颤音构成的连接部分,接入序奏
    \end{center}

    第 199—205 小节中,Sergio Fiorentino 与普列特涅夫均未完全按乐谱处理颤音。Sergio Fiorentino 几乎弹成了如图 3.20 的形式;普列特涅夫除此之外更加“夸张”,将每个颤(震)音独立由慢进入后渐快,仿佛四个声部分离\footnote{在普列特涅夫 2006 年的演奏中可以更加清晰地得到这一点。}:这可以说是富于交响性的演绎。

    \begin{center}
    \includegraphics[scale=0.52]{25.png}\\
    \heiti \fontsize{10.5}{12.6}\selectfont 图 3.20 \ \ 谱例 20:Sergio Fiorentino 对第 199—204 小节的演绎方法转写
    \end{center}


    \section{\heiti \fontsize{16}{19.2}\selectfont 结论}
    受时间、空间和讨论对象的限制,本文既不能够对更多的细节、更多的作品或更多的方面进行分析,也不能直接给出若干条高屋建瓴的演奏意见(所选录音数量不足,分析角度不足)。本文仍有一些尚需改进之处,包括但不限于:
    \begin{enumerate}
        \item 选取的录音不够具代表性。如对 Evgeny Kissin 的演奏,甚至不能称其为当代优秀的演奏;
        \item 缺少理论工具的使用。如和声学角度分析、旋律理论分析等;
        \item 过于依赖描述性形容词的使用。这不利于维持本文严谨性;
        \item 对录音的分析不够全面。如对录音作为一个整体(乐曲作为一个整体)的分析是较少的。
    \end{enumerate}

    在本文的分析中存在对 Evgeny Kissin 的演奏的批评。需要注意的是,这不应当意味着其演绎毫无可取之处,不意味着其演绎是“错误”的\footnote{可能可以说,其演绎与惯常的钢琴演奏美学观点不相吻合之处较多。},也不意味着其他演奏是毫无“瑕疵”的。

    具体地,可以提出一些供未来研究的建议:
    \begin{enumerate}
        \item 找到一种新方式分析以减轻对描述性形容词的依赖;
        \item 挑选曲目、录音时增强针对性。
    \end{enumerate}

    本文以肖邦的两首作品为切入点,在几为演奏者录音比较分析的基础上对演奏肖邦的音乐进行了实践的分析,主要在分句处理、结构分析、音色、多声部处理、动态处理等方面。在录音分析中,已经具体地提出了一些优秀演绎的特点与一些演绎肖邦作品的规律。

    肖邦具有“钢琴诗人”的称号,其钢琴独奏作品,特别是中晚期大型作品具有重要的意义。学习这些作品的乐谱、录音来应用于钢琴演奏的实践中,本文从一个新的角度提出了一些观点;作为演奏者,还有很长的路需要走。





    \newpage
    \begin{appendices}
    \section{\heiti \fontsize{16}{19.2}\selectfont 录音导引}
    \subsection{\heiti \fontsize{15}{18}\selectfont 《f 小调幻想曲》}

    \begin{enumerate}
    \item Jerzy Żurawlew: Muza SX 0075: \href{https://xrdrsp.github.io/uploads/2023/05/01/Żurawlew-49.mp3}{音频地址链接}
    \item Myra Hess: \href{https://xrdrsp.github.io/uploads/2023/05/01/hess-49.mp3}{音频地址链接}
    \item Evgeny Kissin, BMG BVCC-656: \href{https://xrdrsp.github.io/uploads/2023/05/01/kissin-49.mp3}{音频地址链接}
    \end{enumerate}

    \subsection{\heiti \fontsize{15}{18}\selectfont 《降 A 大调幻想波兰舞曲》}

    \begin{enumerate}
    \item Sergio Fiorentino: Saga STM 6021: \href{https://xrdrsp.github.io/uploads/2023/05/01/fiorentino-61.mp3}{音频地址链接}
    \item Marcel Ciampi: \href{https://xrdrsp.github.io/uploads/2023/05/01/ciampi-61.mp3}{音频地址链接}
    \item Mikhail Pletnev, Recital at 2022 Verbier Festival, Switzerland: \href{https://xrdrsp.github.io/uploads/2023/05/01/pletnev-61.mp3}{音频地址链接}
    \end{enumerate}

    \section{\heiti \fontsize{16}{19.2}\selectfont 乐谱导引}
    所有乐谱要么属于共有领域,要么是获得了许可的版权资料 \cite{imslp}。它们均可在 \href{https://imslp.org}{the International Music Score Library Project} 找到。
    % \begin{enumerate}

    \subsection{\heiti \fontsize{15}{18}\selectfont 《f 小调幻想曲》}
        \href{https://xrdrsp.github.io/uploads/2023/05/02/61-ekier.pdf}{这是乐谱地址链接。}
        \begin{longtable}{l l}
            编者 & Carl Mikuli \\ 
            出版商信息 & Complete Works for the Piano, Vol. 6 (\textit{pp.} 75—92)\\
            ~ & New York: G. Schirmer, 1895. Plate 11751.\\
            ~ & Reissue — Vol. 7, 1934. Plate 36395.\\
            版权信息 & Public Domain \cite{49}
        \end{longtable}
    
    \subsection{\heiti \fontsize{15}{18}\selectfont 《降 A 大调幻想波兰舞曲》}
        \href{https://xrdrsp.github.io/uploads/2023/05/02/61-ekier.pdf}{这是乐谱地址链接。}
        \begin{longtable}{l l}
            编者 & Jan Ekier \\
            ~ & Pawe\l{} Kamiński \\
            出版商信息 & National Edition of the Works of Fryderyk Chopin,\\
            ~ & Series A, Vol. VI (\textit{pp.} 76—92)\\
            ~ & Kraków: PWM, 1995. Plate PWM 9386.\\
            版权信息 & Public Domain (not Public Domain in the United States) \cite{61}
        \end{longtable}
    % \end{enumerate}

    \end{appendices}

    \printbibliography[title={参考}]

\end{document}
\documentclass{beamer}
\usepackage[utf8]{inputenc}
\usepackage[T1]{fontenc}
\usepackage{mathabx}
\usepackage{mathpazo}
\usepackage{eulervm}
\usepackage{natbib}
\usepackage{ctex}

\usepackage{multicol}
\usetheme{Ilmenau}
\usefonttheme{default}
\useinnertheme{rounded}

\title{2023 学年第四次社团活动}
\author{徐暄哲、高源\thanks{\href{https://xrdrsp.github.io}{https://xrdrsp.github.io}, \href{mailto:xrdrsp@gmail.com}{xrdrsp@gmail.com}}、徐若钧}
\institute{宁波效实中学(东部校区)模拟联合国社团}
\date{2023 年 11 月 29 日}

\begin{document}

\maketitle

\section{摘要和备注}

\begin{frame}
    \frametitle{摘要和备注}

    这是 2023 学年第四次社团活动的幻灯片,主要内容是模拟联合国基本会场议事规则。

    此次(2023 年 11 月 29 日)的社团课是突然产生的,高源将内容安排得很不完备。抱歉!

    该幻灯片很大程度上受启发于 D2205 张伯伦为以前的社团课制作的课件。其中包含的议事规则内容很大程度上摘自《2023 年北京大学全国中学生模拟联合国大会学术标准手册(修订版)》(简称“北大学标”),以下不再特殊注明。
\end{frame}

\begin{frame}
\frametitle{目录}
\begin{multicols}{2}
\tableofcontents
\end{multicols}
\end{frame}
% \frame{\tableofcontents}

\section{概述}

\subsection{基本概念}

\begin{frame}

\frametitle{模拟联合国(Model United Nations)}

模拟联合国是对联合国及其他相关国际机构进行的学术模仿,依据联合国及其相关机构的运作方式和议事原则,围绕国际热点问题召开的会议。学生们扮演各个国家的外交官参与到会议当中,亲身感受联合国会议的流程和运作方式,如观点阐述、政策辩论等,从而对某个国际问题进行深入探讨。


模拟联合国活动主要通过模拟联合国大会(Model United Nations Conference)开展。一次模拟联合国大会下设若干委员会。


\end{frame}

\begin{frame}{委员会(Committee)和议题(Topic)}

委员会是模拟联合国活动中进行会议学术准备、会议讨论的基本单位,一个委员会有一个或多个对应的议题。一般而言,委员会及其学术设置应为现实中联合国及相关机构的真实体现,议题应该符合现实中该委员会的工作和讨论范畴。

委员会由主席团、代表、观察员和会场志愿者组成。

如,上一次社团活动的模拟会场中,“委员会”指“联合国安理会”,“议题”可能指“巴以冲突”等。

\end{frame}

\begin{frame}{主席团(Dais)}

主席团是委员会的管理团队,应对会议进行主持、组织和指导。主席团不应参与会议讨论或领导会议进程。主席团包括会议指导、主席与主席助理。

\end{frame}

\begin{frame}{代表(Delegate)}

代表在会议中扮演特定角色参与相应委员会的讨论。

主席团可以在代表席位中设置\textbf{观察国}。观察国并非委员会成员国,但因与本委员会议题关系紧密等原因获邀参与委员会会议。一般而言,观察国与本委员会成员国行使同等权利并具有同等义务;但是,观察国无权起草决议草案、修正案和指令草案及其修正案,观察国也没有实质性投票权力。主席团可以根据委员会需要确定观察国权利和义务。

\end{frame}

\begin{frame}{席位(Seat)}

席位是代表在委员会中所扮演的角色。大多数情况下,席位是主权国家;在一些情况下,委员会将设置国际组织、企业等多种形式的席位。

代表在正式会期中的一切行动应该\textbf{基于所代表的席位立场完成},这要求代表在会前对席位有充分的了解。

\end{frame}

\begin{frame}{单代表制(Single Delegation)和双代表制(Double Delegation)}

单代表制指一位代表扮演一个席位角色,双代表制指两位代表共同扮演一个席位角色。除此之外,两者在规则上没有区别。

一般而言,双代表制的委员会对会前准备的要求更高,两位代表需要合理分工,共同完成会前和会上有关工作。

\end{frame}

\begin{frame}{国家牌(Placard)}
国家牌是代表出席会议和采取行动的唯一凭证。代表未举起国家牌即采取行动,主席团应视为无效。

\end{frame}

\begin{frame}{背景文件(Background Guide)}
    
背景文件是主席团在会前撰写的学术性文件,旨在帮助代表梳理委员会和议题的基本情况、发展脉络,并促进代表完成更进一步的学术准备。

背景文件是代表完成学术准备的基础,代表应该在会前认真完整阅读,并自行搜集其他资料。

对于与杏鸣模拟联合国大会和效实新生会规模类似的会议,背景文件一般于报名结束后,会议开始前一个月左右下发。

参考:2023 年北京大学全国中学生模拟联合国大会联合国人权理事会会场的背景文件、2023 年效实中学新生会英文场背景文件。

\end{frame}

\begin{frame}{省略的基本概念}

观察员、志愿者、工作语言、座次、代表牌、意向条、背景更新……

“北大学标”中详细介绍了。

\end{frame}


\section{会前准备}

\begin{frame}{学术测验(Academic Test)}

各个委员会将按照自身的会前准备特点、需求布置具体作业(测验)内容,代表需要按照相应的要求按时完成、提交。

对于与杏鸣模拟联合国大会和效实新生会规模类似的会议,学术测验一般于报名结束后,会议开始前一个月左右下发。
    
\end{frame}

\begin{frame}{立场文件(Position Paper)}

立场文件是代表在会前撰写的有关本席位在委员会议题 的立场的重要学术性文件,旨在帮助代表在写作过程中代表梳理有关内容,也能为其他代表提供了解本席位\textbf{基本信息和态度}的机会。

充足的会前准备能够很大程度上提高参会体验。

立场文件写作要求与会议文件写作要求类似,下面再展开。

参考:“北大学标”第 12 页。

\end{frame}

\section{会议流程}

\begin{frame}{示例}

北京大学全国中学生模拟联合国大会一般安排七次分组会议。其中,第一、四次分组会议是晚间会期,时长各为 180 分钟;第二、三、五、六、七次分组会议是日间会期,时长各为 210 分钟。会议总时长约为 1410 分钟,约 23 小时。
大会的常规委员会将分别经过:\textbf{各国代表阐述本国立场——开始形成国家集团——开始撰写工作文件——在工作文件的基础上进一步讨论——国家集团基本确定——开始撰写决议草案——在决议草案的基础上进一步讨论——投票表决},共八个阶段。这八个阶段始终在七次分组会议的“正式辩论”和“非正式辩论”中交替进行。

\end{frame}

\begin{frame}{示例}
    \begin{multicols}{2}
    \begin{center}\includegraphics[scale=0.08]{1.jpg}\end{center}\begin{tiny}\begin{itemize}\item[图 1] 2023 年效实新生会的日程安排\end{itemize}\end{tiny}
    \end{multicols}
\end{frame}

\section{常规委员会议事规则}

\subsection{会议开始和正式辩论}

\begin{frame}{点名}

点名在会议开始及每个分组会议开始时进行。首先,主席召集代表就座,并宣布分组会议开始。之后主席宣布开始点名。点名按照席位汉语拼音首字母(或英文名首字母) A 到 Z 的顺序进行。代表在被点到时,应高举国家牌并答“出席”(“Present”)。点名后,主席助理应宣布下列事由:实际出席数、是否符合法定出席数,会议的简单多数(Simple Majority)、三分之二多数(或“绝对多数”)(Two-Thirds Majority)和百分之二十数(20\% of the Number)。

\end{frame}

\begin{frame}{详细信息}

以下数目的约定与动议、文件等投票有关:

\begin{itemize}
\item 法定出席数是会场应出席人数的简单多数。
\item 简单多数是实际出席数乘以二分之一、向下取整、加一的数目。
\item 三分之二多数(绝对多数)是实际出席数乘以二、除以三、向上取整的数目。
\item 百分之二十数是实际出席数乘以百分之二十、向上取整的数目。
\end{itemize}

\end{frame}

\begin{frame}{进入正式辩论}

正式辩论是在终止辩论的动议被通过且所有非友好修正案及决议草案都被投票之前,会议围绕既定议题开展的活动。
点名结束后,主发言名单自动开启,会议进入正式辩论。

\end{frame}

\begin{frame}{主发言名单(the Speakers’ List)}

主席需要请希望加入主发言名单的代表高举国家牌,主席随机点出,并由主席助理记录名单及顺序。记录完毕后,在主发言名单的代表按照顺序发言。主发言名单的默认发言时间是 120 秒。这一时间可以由\textbf{动议}修改。

主发言名单将\textbf{一直保留到会议全部结束},即,主发言名单在不同分组会议之间延续,不需要重新设置。

一般而言,连续三个动议未获得通过时,\textbf{自动}返回主发言名单。

主发言名单结束意味着正式辩论结束,也意味着会议结束。

\end{frame}

\begin{frame}{主发言名单的时间让渡(Yield)}

如果代表发言结束后的剩余时间大于或等于 10 秒,代表可以选择让渡。方式包括:
\begin{itemize}
\item 让渡给主席(Yield to Chair):主席自行处理剩余时间;
\item 让渡给代表(Yield to the delegate of ...):被让渡的代表获得剩余的发言时间,发言结束后不可再次让渡;
\item 让渡给问题(Yield to question):主席请希望提问的代表高举国家牌,主席随机点出一位,该代表进行提问(提问不占用时间),发言代表使用剩余的时间对问题进行回答;
\item 让渡给评论(Yield to comment):主席请希望评论的代表高举国家牌,主席随机点出一位,该代表获得剩余的时间对刚才的发言进行评论。
\end{itemize}

\end{frame}

\begin{frame}{正式辩论的暂停和结束}

正式辩论在下列情况下暂停或结束:

\begin{enumerate}
\item 会议通过\textbf{有主持核心磋商或自由磋商的动议}时,正式辩论暂停;
\item 代表举牌\textbf{提出问题}时,正式辩论暂停;
\item 会议通过\textbf{暂时休会的动议}时,正式辩论暂停;
\item 大会组委会或主席团认为必要时,正式辩论暂停;
\item 会议通过\textbf{结束辩论动议}时,正式辩论结束;
\item 正式辩论发言名单(主发言名单)中的最后一位代表发言结束,并且没有代表希望继续发言时,正式辩论自动结束。
\end{enumerate}

\end{frame}

\end{document}
